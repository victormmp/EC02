\documentclass[]{article}
\usepackage{lmodern}
\usepackage{amssymb,amsmath}
\usepackage{ifxetex,ifluatex}
\usepackage{fixltx2e} % provides \textsubscript
\ifnum 0\ifxetex 1\fi\ifluatex 1\fi=0 % if pdftex
  \usepackage[T1]{fontenc}
  \usepackage[utf8]{inputenc}
\else % if luatex or xelatex
  \ifxetex
    \usepackage{mathspec}
  \else
    \usepackage{fontspec}
  \fi
  \defaultfontfeatures{Ligatures=TeX,Scale=MatchLowercase}
\fi
% use upquote if available, for straight quotes in verbatim environments
\IfFileExists{upquote.sty}{\usepackage{upquote}}{}
% use microtype if available
\IfFileExists{microtype.sty}{%
\usepackage{microtype}
\UseMicrotypeSet[protrusion]{basicmath} % disable protrusion for tt fonts
}{}
\usepackage[margin=1in]{geometry}
\usepackage{hyperref}
\hypersetup{unicode=true,
            pdftitle={Estudo de Caso 2: Planejamento e Análise de Experimentos},
            pdfauthor={Matheus Marzochi, Mayra Mota, Rafael Ramos e Victor Magalhães},
            pdfborder={0 0 0},
            breaklinks=true}
\urlstyle{same}  % don't use monospace font for urls
\usepackage{color}
\usepackage{fancyvrb}
\newcommand{\VerbBar}{|}
\newcommand{\VERB}{\Verb[commandchars=\\\{\}]}
\DefineVerbatimEnvironment{Highlighting}{Verbatim}{commandchars=\\\{\}}
% Add ',fontsize=\small' for more characters per line
\usepackage{framed}
\definecolor{shadecolor}{RGB}{248,248,248}
\newenvironment{Shaded}{\begin{snugshade}}{\end{snugshade}}
\newcommand{\AlertTok}[1]{\textcolor[rgb]{0.94,0.16,0.16}{#1}}
\newcommand{\AnnotationTok}[1]{\textcolor[rgb]{0.56,0.35,0.01}{\textbf{\textit{#1}}}}
\newcommand{\AttributeTok}[1]{\textcolor[rgb]{0.77,0.63,0.00}{#1}}
\newcommand{\BaseNTok}[1]{\textcolor[rgb]{0.00,0.00,0.81}{#1}}
\newcommand{\BuiltInTok}[1]{#1}
\newcommand{\CharTok}[1]{\textcolor[rgb]{0.31,0.60,0.02}{#1}}
\newcommand{\CommentTok}[1]{\textcolor[rgb]{0.56,0.35,0.01}{\textit{#1}}}
\newcommand{\CommentVarTok}[1]{\textcolor[rgb]{0.56,0.35,0.01}{\textbf{\textit{#1}}}}
\newcommand{\ConstantTok}[1]{\textcolor[rgb]{0.00,0.00,0.00}{#1}}
\newcommand{\ControlFlowTok}[1]{\textcolor[rgb]{0.13,0.29,0.53}{\textbf{#1}}}
\newcommand{\DataTypeTok}[1]{\textcolor[rgb]{0.13,0.29,0.53}{#1}}
\newcommand{\DecValTok}[1]{\textcolor[rgb]{0.00,0.00,0.81}{#1}}
\newcommand{\DocumentationTok}[1]{\textcolor[rgb]{0.56,0.35,0.01}{\textbf{\textit{#1}}}}
\newcommand{\ErrorTok}[1]{\textcolor[rgb]{0.64,0.00,0.00}{\textbf{#1}}}
\newcommand{\ExtensionTok}[1]{#1}
\newcommand{\FloatTok}[1]{\textcolor[rgb]{0.00,0.00,0.81}{#1}}
\newcommand{\FunctionTok}[1]{\textcolor[rgb]{0.00,0.00,0.00}{#1}}
\newcommand{\ImportTok}[1]{#1}
\newcommand{\InformationTok}[1]{\textcolor[rgb]{0.56,0.35,0.01}{\textbf{\textit{#1}}}}
\newcommand{\KeywordTok}[1]{\textcolor[rgb]{0.13,0.29,0.53}{\textbf{#1}}}
\newcommand{\NormalTok}[1]{#1}
\newcommand{\OperatorTok}[1]{\textcolor[rgb]{0.81,0.36,0.00}{\textbf{#1}}}
\newcommand{\OtherTok}[1]{\textcolor[rgb]{0.56,0.35,0.01}{#1}}
\newcommand{\PreprocessorTok}[1]{\textcolor[rgb]{0.56,0.35,0.01}{\textit{#1}}}
\newcommand{\RegionMarkerTok}[1]{#1}
\newcommand{\SpecialCharTok}[1]{\textcolor[rgb]{0.00,0.00,0.00}{#1}}
\newcommand{\SpecialStringTok}[1]{\textcolor[rgb]{0.31,0.60,0.02}{#1}}
\newcommand{\StringTok}[1]{\textcolor[rgb]{0.31,0.60,0.02}{#1}}
\newcommand{\VariableTok}[1]{\textcolor[rgb]{0.00,0.00,0.00}{#1}}
\newcommand{\VerbatimStringTok}[1]{\textcolor[rgb]{0.31,0.60,0.02}{#1}}
\newcommand{\WarningTok}[1]{\textcolor[rgb]{0.56,0.35,0.01}{\textbf{\textit{#1}}}}
\usepackage{graphicx,grffile}
\makeatletter
\def\maxwidth{\ifdim\Gin@nat@width>\linewidth\linewidth\else\Gin@nat@width\fi}
\def\maxheight{\ifdim\Gin@nat@height>\textheight\textheight\else\Gin@nat@height\fi}
\makeatother
% Scale images if necessary, so that they will not overflow the page
% margins by default, and it is still possible to overwrite the defaults
% using explicit options in \includegraphics[width, height, ...]{}
\setkeys{Gin}{width=\maxwidth,height=\maxheight,keepaspectratio}
\IfFileExists{parskip.sty}{%
\usepackage{parskip}
}{% else
\setlength{\parindent}{0pt}
\setlength{\parskip}{6pt plus 2pt minus 1pt}
}
\setlength{\emergencystretch}{3em}  % prevent overfull lines
\providecommand{\tightlist}{%
  \setlength{\itemsep}{0pt}\setlength{\parskip}{0pt}}
\setcounter{secnumdepth}{0}
% Redefines (sub)paragraphs to behave more like sections
\ifx\paragraph\undefined\else
\let\oldparagraph\paragraph
\renewcommand{\paragraph}[1]{\oldparagraph{#1}\mbox{}}
\fi
\ifx\subparagraph\undefined\else
\let\oldsubparagraph\subparagraph
\renewcommand{\subparagraph}[1]{\oldsubparagraph{#1}\mbox{}}
\fi

%%% Use protect on footnotes to avoid problems with footnotes in titles
\let\rmarkdownfootnote\footnote%
\def\footnote{\protect\rmarkdownfootnote}

%%% Change title format to be more compact
\usepackage{titling}

% Create subtitle command for use in maketitle
\providecommand{\subtitle}[1]{
  \posttitle{
    \begin{center}\large#1\end{center}
    }
}

\setlength{\droptitle}{-2em}

  \title{Estudo de Caso 2: Planejamento e Análise de Experimentos}
    \pretitle{\vspace{\droptitle}\centering\huge}
  \posttitle{\par}
    \author{Matheus Marzochi, Mayra Mota, Rafael Ramos e Victor Magalhães}
    \preauthor{\centering\large\emph}
  \postauthor{\par}
      \predate{\centering\large\emph}
  \postdate{\par}
    \date{30 de Setembro de 2019}


\begin{document}
\maketitle

\begin{verbatim}
## Downloading GitHub repo dgrtwo/broom@master
\end{verbatim}

\begin{verbatim}
## Installing 9 packages: dplyr, ellipsis, generics, tidyr, pkgconfig, tidyselect, plogr, lifecycle, digest
\end{verbatim}

\begin{verbatim}
## Installing packages into '/home/victormmp/R/x86_64-pc-linux-gnu-library/3.6'
## (as 'lib' is unspecified)
\end{verbatim}

\begin{verbatim}
## Installing package into '/home/victormmp/R/x86_64-pc-linux-gnu-library/3.6'
## (as 'lib' is unspecified)
## Installing package into '/home/victormmp/R/x86_64-pc-linux-gnu-library/3.6'
## (as 'lib' is unspecified)
\end{verbatim}

\begin{verbatim}
## Loading required package: magrittr
\end{verbatim}

\hypertarget{resumo}{%
\subsection{Resumo}\label{resumo}}

O experimento disponibilizado para este trabalho consiste em avaliar se
o estilo de vida entre duas populações de estudantes se alterou ao longo
do tempo. Nessa avaliação duas amostras foram utilizadas, uma contendo
informações (peso e altura dos alunos) referentes ao ano de 2016 e a
outra com informações referentes a 2017.

Neste trabalho o IMC (Índice de Massa Corporal) é usado como parâmetro
de avaliação do estilo de vida. Vale ressaltar que esse indicador possui
limitações como pode ser verificado em {[}1{]}{[}2{]}. Além disso, as
duas populações foram subdivididas por gênero. A motivação desta divisão
é a consideração de que, em média, o IMC masculino pode ser diferente do
feminino.

\hypertarget{papeis-desempenhados}{%
\subsection{Papéis Desempenhados}\label{papeis-desempenhados}}

A divisão de tarefas no grupo segue a descrição da \emph{Declaração de
Políticas de Equipe}. Estando aqui organizada da seguinte forma:

\begin{itemize}
\tightlist
\item
  Matheus: Verificador
\item
  Mayra: Monitora
\item
  Rafael: Coordenador
\item
  Victor: Revisor
\end{itemize}

\hypertarget{planejamento-do-experimento}{%
\subsection{Planejamento do
Experimento}\label{planejamento-do-experimento}}

A hipótese nula (\(H_{0}\)) usada neste experimento é de que a diferença
das médias de IMC das duas populações é nula. A hipótese alternativa
(\(H_{1}\)), por sua vez, afirma que existe sim uma alteração entre as
médias dos dois semestres.

\[
\begin{cases}
  H_0: \mu_{2016/2} - \mu_{2017/2} = 0 \\
  H_1: \mu_{2016/2} - \mu_{2017/2} \neq 0
\end{cases}
\]

Como a comparação é relativa a duas amostras, em função da influência do
gênero no valor do IMC, as análises são realizadas de forma independente
para cada sexo. Como consequência dessa independência, neste trabalho
são realizados testes de comparação simples entre as populações.

\hypertarget{coleta-dos-dados}{%
\subsection{Coleta dos Dados}\label{coleta-dos-dados}}

O procedimento de coleta de dados foi baseado na rotina presente abaixo.

\begin{Shaded}
\begin{Highlighting}[]
\NormalTok{data_}\DecValTok{2017}\NormalTok{ <-}\StringTok{ }\KeywordTok{read.csv}\NormalTok{(}\DataTypeTok{file=}\StringTok{'CS01_20172.csv'}\NormalTok{, }\DataTypeTok{sep=}\StringTok{';'}\NormalTok{)}\CommentTok{#Dados ano 2016}
\NormalTok{data_}\DecValTok{2016}\NormalTok{<-}\KeywordTok{read.csv}\NormalTok{(}\StringTok{"imc_20162.csv"}\NormalTok{);}\CommentTok{#Dados ano 2017}

\NormalTok{PPGEE_dados=data_}\DecValTok{2016}\NormalTok{[data_}\DecValTok{2016}\NormalTok{[}\DecValTok{2}\NormalTok{]}\OperatorTok{==}\StringTok{'PPGEE'}\NormalTok{,];}

\CommentTok{#Dados população feminina e masculina do ano de 2016.}
\NormalTok{Dados_Masculino_}\DecValTok{2016}\NormalTok{=PPGEE_dados[PPGEE_dados[}\DecValTok{3}\NormalTok{]}\OperatorTok{==}\StringTok{'M'}\NormalTok{,];}
\NormalTok{Dados_Feminino_}\DecValTok{2016}\NormalTok{=PPGEE_dados[PPGEE_dados[}\DecValTok{3}\NormalTok{]}\OperatorTok{==}\StringTok{'F'}\NormalTok{,];}

\NormalTok{Height_Masc_}\DecValTok{2016}\NormalTok{=Dados_Masculino_}\DecValTok{2016}\NormalTok{[,}\DecValTok{4}\NormalTok{];}
\NormalTok{Heigh_Feminino_}\DecValTok{2016}\NormalTok{=Dados_Feminino_}\DecValTok{2016}\NormalTok{[,}\DecValTok{4}\NormalTok{];}
\NormalTok{Weight_Masculino_}\DecValTok{2016}\NormalTok{=Dados_Masculino_}\DecValTok{2016}\NormalTok{[,}\DecValTok{5}\NormalTok{];}
\NormalTok{Weight_Feminino_}\DecValTok{2016}\NormalTok{=Dados_Feminino_}\DecValTok{2016}\NormalTok{[,}\DecValTok{5}\NormalTok{];}

\CommentTok{#Calculo do IMC masculino 2016}
\NormalTok{IMC_masculino_}\DecValTok{2016}\NormalTok{=(Weight_Masculino_}\DecValTok{2016}\OperatorTok{/}\NormalTok{((Height_Masc_}\DecValTok{2016}\NormalTok{)}\OperatorTok{*}\NormalTok{(Height_Masc_}\DecValTok{2016}\NormalTok{)));}
\CommentTok{#Calculo do IMC feminino 2016}
\NormalTok{IMC_Feminino_}\DecValTok{2016}\NormalTok{=(Weight_Feminino_}\DecValTok{2016}\OperatorTok{/}\NormalTok{((Heigh_Feminino_}\DecValTok{2016}\NormalTok{)}\OperatorTok{*}\NormalTok{(Heigh_Feminino_}\DecValTok{2016}\NormalTok{)));}


\CommentTok{#Dados população feminina e masculina do ano de 2017.}
\NormalTok{Dados_Masculino_}\DecValTok{2017}\NormalTok{=data_}\DecValTok{2017}\NormalTok{[data_}\DecValTok{2017}\NormalTok{[}\DecValTok{3}\NormalTok{]}\OperatorTok{==}\StringTok{'M'}\NormalTok{,];}
\NormalTok{Dados_Feminino_}\DecValTok{2017}\NormalTok{=data_}\DecValTok{2017}\NormalTok{[data_}\DecValTok{2017}\NormalTok{[}\DecValTok{3}\NormalTok{]}\OperatorTok{==}\StringTok{'F'}\NormalTok{,];}

\NormalTok{Height_Masculino_}\DecValTok{2017}\NormalTok{=Dados_Masculino_}\DecValTok{2017}\NormalTok{[,}\DecValTok{2}\NormalTok{];}
\NormalTok{Height_Feminino_}\DecValTok{2017}\NormalTok{=Dados_Feminino_}\DecValTok{2017}\NormalTok{[,}\DecValTok{2}\NormalTok{];}
\NormalTok{Weight_Masc_}\DecValTok{2017}\NormalTok{=Dados_Masculino_}\DecValTok{2017}\NormalTok{[,}\DecValTok{1}\NormalTok{];}
\NormalTok{Weight_Feminino_}\DecValTok{2017}\NormalTok{=Dados_Feminino_}\DecValTok{2017}\NormalTok{[,}\DecValTok{1}\NormalTok{];}

\CommentTok{#Calculo do IMC masculino 2017}
\NormalTok{IMC_masculino_}\DecValTok{2017}\NormalTok{=(Weight_Masc_}\DecValTok{2017}\OperatorTok{/}\NormalTok{((Height_Masculino_}\DecValTok{2017}\NormalTok{)}\OperatorTok{*}\NormalTok{(Height_Masculino_}\DecValTok{2017}\NormalTok{)));}
\CommentTok{#Calculo do IMC feminino 2017}
\NormalTok{IMC_Feminino_}\DecValTok{2017}\NormalTok{=(Weight_Feminino_}\DecValTok{2017}\OperatorTok{/}\NormalTok{((Height_Feminino_}\DecValTok{2017}\NormalTok{)}\OperatorTok{*}\NormalTok{(Height_Feminino_}\DecValTok{2017}\NormalTok{)));}
\end{Highlighting}
\end{Shaded}

\hypertarget{analise-exploratoria-dos-dados}{%
\subsection{Análise Exploratória dos
Dados}\label{analise-exploratoria-dos-dados}}

Antes da análise estatística, os dados são avaliados de forma
qualitativa na análise exploratória com o objetivo de se extrair
informações úteis. Entre as ferramentas usadas para se explorar os dados
existentes está o Histograma.

Segue abaixo o trecho de código usado para gerar o histograma da
população masculina dos anos de 2016 e 2017.

\begin{Shaded}
\begin{Highlighting}[]
\NormalTok{p1 <-}\StringTok{ }\KeywordTok{ggplot}\NormalTok{(}\KeywordTok{as.data.frame}\NormalTok{(IMC_masculino_}\DecValTok{2016}\NormalTok{), }\KeywordTok{aes}\NormalTok{(}\DataTypeTok{x =}\NormalTok{ IMC_masculino_}\DecValTok{2016}\NormalTok{))}
\NormalTok{p1 <-}\StringTok{ }\NormalTok{p1 }\OperatorTok{+}\KeywordTok{geom_histogram}\NormalTok{(}\DataTypeTok{bins =} \DecValTok{15}\NormalTok{)}\OperatorTok{+}\KeywordTok{xlab}\NormalTok{(}\StringTok{"IMC Masculino 2016"}\NormalTok{)}\OperatorTok{+}\KeywordTok{ylab}\NormalTok{(}\StringTok{"Nº de observações"}\NormalTok{)}
\NormalTok{p2 <-}\StringTok{ }\KeywordTok{ggplot}\NormalTok{(}\KeywordTok{as.data.frame}\NormalTok{(IMC_masculino_}\DecValTok{2017}\NormalTok{), }\KeywordTok{aes}\NormalTok{(}\DataTypeTok{x =}\NormalTok{ IMC_masculino_}\DecValTok{2017}\NormalTok{))}
\NormalTok{p2 <-}\StringTok{ }\NormalTok{p2 }\OperatorTok{+}\StringTok{ }\KeywordTok{geom_histogram}\NormalTok{(}\DataTypeTok{bins =} \DecValTok{15}\NormalTok{)}\OperatorTok{+}\KeywordTok{xlab}\NormalTok{(}\StringTok{"IMC Masculino 2017"}\NormalTok{)}\OperatorTok{+}\KeywordTok{ylab}\NormalTok{(}\StringTok{"Nº de observações"}\NormalTok{)}
\KeywordTok{ggarrange}\NormalTok{(p1, p2, }\DataTypeTok{nrow =} \DecValTok{1}\NormalTok{, }\DataTypeTok{ncol =} \DecValTok{2}\NormalTok{)}
\end{Highlighting}
\end{Shaded}

\includegraphics{report_ec2_files/figure-latex/histograma Masculino-1.pdf}

Segue abaixo o trecho de código usado para gerar o histograma da
população feminina dos anos de 2016 e 2017.

\begin{Shaded}
\begin{Highlighting}[]
\NormalTok{p1 <-}\StringTok{ }\KeywordTok{ggplot}\NormalTok{(}\KeywordTok{as.data.frame}\NormalTok{(IMC_Feminino_}\DecValTok{2016}\NormalTok{), }\KeywordTok{aes}\NormalTok{(}\DataTypeTok{x =}\NormalTok{ IMC_Feminino_}\DecValTok{2016}\NormalTok{))}
\NormalTok{p1 <-}\StringTok{ }\NormalTok{p1 }\OperatorTok{+}\KeywordTok{geom_histogram}\NormalTok{(}\DataTypeTok{bins =} \DecValTok{8}\NormalTok{)}\OperatorTok{+}\KeywordTok{xlab}\NormalTok{(}\StringTok{"IMC Feminino 2016"}\NormalTok{)}\OperatorTok{+}\KeywordTok{ylab}\NormalTok{(}\StringTok{"Nº de observações"}\NormalTok{)}
\NormalTok{p2 <-}\StringTok{ }\KeywordTok{ggplot}\NormalTok{(}\KeywordTok{as.data.frame}\NormalTok{(IMC_Feminino_}\DecValTok{2017}\NormalTok{), }\KeywordTok{aes}\NormalTok{(}\DataTypeTok{x =}\NormalTok{ IMC_Feminino_}\DecValTok{2017}\NormalTok{))}
\NormalTok{p2 <-}\StringTok{ }\NormalTok{p2 }\OperatorTok{+}\StringTok{ }\KeywordTok{geom_histogram}\NormalTok{(}\DataTypeTok{bins =} \DecValTok{8}\NormalTok{)}\OperatorTok{+}\KeywordTok{xlab}\NormalTok{(}\StringTok{"IMC Feminino 2017"}\NormalTok{)}\OperatorTok{+}\KeywordTok{ylab}\NormalTok{(}\StringTok{"Nº de observações"}\NormalTok{)}
\KeywordTok{ggarrange}\NormalTok{(p1, p2, }\DataTypeTok{nrow =} \DecValTok{1}\NormalTok{, }\DataTypeTok{ncol =} \DecValTok{2}\NormalTok{)}
\end{Highlighting}
\end{Shaded}

\includegraphics{report_ec2_files/figure-latex/histograma Feminino-1.pdf}
Através da análise exploratória dos dados é possível observar que o
número de amostras fornecidos é pequeno, especialmente no caso feminino.
Para a população masculina apesar do número de amostras não ser elevado,
a distribuição das amostras leva há indícios que a distribuição pode ser
normal. No caso feminino há indícios de que seja necessário o uso de
estatística não paramétrica. Essas hipótes são analisadas mais
profundamente na análise estatística.

\hypertarget{analise-estatistica}{%
\subsection{Análise estatística}\label{analise-estatistica}}

\hypertarget{conclusoes}{%
\subsection{Conclusões}\label{conclusoes}}

\hypertarget{bibliografia}{%
\subsection{Bibliografia}\label{bibliografia}}

{[}1{]}
\url{https://www.nytimes.com/interactive/projects/cp/summer-of-science-2015/latest/how-often-is-bmi-misleading}

{[}2{]}
\url{https://science.sciencemag.org/content/341/6148/856.summary}


\end{document}
